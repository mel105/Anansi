\section{ECEF $\rightarrow$ ENU $\&$ ENU $\rightarrow$ ECEF}

Predpokládajme, že v tomto príklade uvažovaný rotačný elipsoid (napríklad WGS-84 alebo GRS-80) je geocentrický, to znamená, že stred elipsoidu sa nachádza v strede zemského telesa, potom transformácia súradníc medzi zemským geocentrickým systémom súradníc (xyz) a lokálnym topocentrickým (alebo tiež lokálnym geodetickým - enu) môže byť vyjadrená predpisom \cite{Soler1998}

\begin{equation}
\begin{bmatrix}
e \\
n \\
u
\end{bmatrix} = 
\mathbf{C}_{enu}^{xyz}
\begin{bmatrix}
x \\
y \\
z
\end{bmatrix}.
\label{rov:ecef2enu1}
\end{equation}

Pre vyjadrenie tranformácie medzi uvedenými systémami si potrebujeme vyjadriť transformačnú maticu, v tomto prípade tzv. rotačnú maticu. Vychádzajme z rovnice \ref{rov:generRotMat}, zostavíme rotačnú maticu pre rotáciu v priestore a to pomocou jednoduchých rotácii v každej osi samostatne.

Rotačná matica okolo osi \textit{z} v smere hodinových ručičiek nadobúdne tvar
\begin{equation}
\mathbf{R_{1}}\left(\theta\right) = 
\begin{bmatrix}
\cos{\left(\theta\right)} & \sin{\left(\theta\right)} & 0 \\
-\sin{\left(\theta\right)} & \cos{\left(\theta\right)} & 0 \\
0 & 0 & 1
\end{bmatrix},
\end{equation}
pričom rotácia okolo osi \textit{z} je $\cos{\left(\theta_{z,w}\right)} = 1$,

% pretože uhol medzi osami \textit{z} a \textit{w} je rovný nule a cosínus nuly je rovný jedna. Ďalej platí, že kosínus uhla $\cos{\left(\theta_{z,u}\right)} = 0$, pretože uhol medzi osami \textit{z} a \textit{u} je rovný 90 stupňov a kosínus tohto uhla je teda nula. Rovnako tento predpoklad platí aj pre $\cos{\left(\theta_{z,v}\right)}$, $\cos{\left(\theta_{x,w}\right)}$ a $\cos{\left(\theta_{y,w}\right)}$.



%
%
%Vyjadrenie transformačnej matice $\mathbf{C}_{enu}^{xyz}$ medzi dvoma pravouhlými karteziánskými súradnými systémami pozostáva z dvoch rotácii, konrkétne:
%\begin{enumerate}
%\item rotácie okolo osi \textit{z} o uhol $\pi/2 + \lambda$ a 
%\item rotácie okolo osi \textit{y} o uhol $\pi/2 - \varphi$.
%\end{enumerate}
%
%
%Potom transformačná matica, v tomto prípade rotačná matica nadobudne tvaru
%\begin{equation}
%
%\end{equation}