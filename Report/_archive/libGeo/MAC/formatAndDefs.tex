\usepackage{ucs}                                 % Kodovani unicode
\usepackage[utf8x]{inputenc}                     % Kodovan� UTF-8
\usepackage{textcomp}
\usepackage[slovak]{babel}                       % Podpora slovenciny
\usepackage[T1]{fontenc}                         % Ciarky a podobne

\usepackage[left=1.5in,right=1.3in,top=1.1in,bottom=1.1in,includefoot,includehead,headheight=13.6pt]{geometry}

\usepackage{longtable}
\usepackage{tabularx}
\usepackage{exercise,chngcntr}
\usepackage{enumerate}
\usepackage{amsmath,amssymb}
\usepackage{caption}
\usepackage{subcaption}
\usepackage[outercaption]{sidecap}
\counterwithin{Exercise}{chapter}

\renewcommand{\baselinestretch}{0.95}

%\DeclareUnicodeCharacter{FB01}{fi}
%\usepackage{bbding}                         % pre checkmark
%\usepackage{subfig}
%\captionsetup[subfigure]{position=top, singlelinecheck=false} % other options margin=0pt, parskip=0pt,

% The chapters stype; Sonny, Lenny, Glenn, Conny, Rejne and Bjarne = nie je dobra pre iny jazyk ja anglictina!!.

\usepackage[Conny]{fncychap}

% Table of contents for each chapter

\usepackage[nottoc, notlof, notlot]{tocbibind}
\usepackage{minitoc}
\setcounter{minitocdepth}{2}
\mtcindent=15pt
% Use \minitoc where to put a table of contents

\usepackage{aecompl}

% Glossary / list of abbreviations

%\usepackage[intoc]{nomencl}
\usepackage[refpage]{nomencl}
\renewcommand{\nomname}{List of Abbreviations}

\makenomenclature

% make Glossaries
\usepackage[acronym]{glossaries}

\makeglossaries

% My pdf code

\usepackage{ifpdf}

\ifpdf
  \usepackage[pdftex]{graphicx}
  \DeclareGraphicsExtensions{.jpg}
  \usepackage[a4paper,pagebackref,hyperindex=true]{hyperref}
\else
  \usepackage{graphicx}
  \DeclareGraphicsExtensions{.ps,.eps}
  \usepackage[a4paper,dvipdfm,pagebackref,hyperindex=true]{hyperref}
\fi

\graphicspath{{.}{images/}}

% nicer backref links
\renewcommand*{\backref}[1]{}
\renewcommand*{\backrefalt}[4]{%
\ifcase #1 %
(Necitovan\'{a}.)%
\or
(Citovan\'{a} na strane~#2.)%
\else
(Citovan\'{a} na stran\'{a}ch~#2.)%
\fi}
\renewcommand*{\backrefsep}{, }
\renewcommand*{\backreftwosep}{ a~}
\renewcommand*{\backreflastsep}{ a~}

% Links in pdf
\usepackage{color}
\definecolor{linkcol}{rgb}{0,0,0.4} 
\definecolor{citecol}{rgb}{0.5,0,0} 
\definecolor{mygray}{gray}{0.6}
% Change this to change the informations included in the pdf file

% See hyperref documentation for information on those parameters

\hypersetup
{
bookmarksopen=true,
pdftitle="Thesis",
pdfauthor="Michal ELIAS", 
pdfsubject="tropospheric delay corrections calculation", %subject of the document
%pdftoolbar=false, % toolbar hidden
pdfmenubar=true, %menubar shown
pdfhighlight=/O, %effect of clicking on a link
colorlinks=true, %couleurs sur les liens hypertextes
pdfpagemode=None, %aucun mode de page
pdfpagelayout=SinglePage, %ouverture en simple page
pdffitwindow=true, %pages ouvertes entierement dans toute la fenetre
linkcolor=linkcol, %couleur des liens hypertextes internes
citecolor=citecol, %couleur des liens pour les citations
urlcolor=linkcol %couleur des liens pour les url
}

% definitions.
% -------------------

\setcounter{secnumdepth}{3}
\setcounter{tocdepth}{2}

% Some useful commands and shortcut for maths:  partial derivative and stuff

\newcommand{\pd}[2]{\frac{\partial #1}{\partial #2}}
\def\abs{\operatorname{abs}}
\def\argmax{\operatornamewithlimits{arg\,max}}
\def\argmin{\operatornamewithlimits{arg\,min}}
\def\diag{\operatorname{Diag}}
\newcommand{\eqRef}[1]{(\ref{#1})}

\usepackage{rotating}                    % Sideways of figures & tables
%\usepackage{bibunits}
%\usepackage[sectionbib]{chapterbib}          % Cross-reference package (Natural BiB)
%\usepackage{natbib}                  % Put References at the end of each chapter
                                         % Do not put 'sectionbib' option here.
                                         % Sectionbib option in 'natbib' will do.
\usepackage{fancyhdr}                    % Fancy Header and Footer

% \usepackage{txfonts}                     % Public Times New Roman text & math font
  
%%% Fancy Header %%%%%%%%%%%%%%%%%%%%%%%%%%%%%%%%%%%%%%%%%%%%%%%%%%%%%%%%%%%%%%%%%%
% Fancy Header Style Options

\pagestyle{fancy}                       % Sets fancy header and footer
\fancyfoot{}                            % Delete current footer settings

%\renewcommand{\chaptermark}[1]{         % Lower Case Chapter marker style
%  \markboth{\chaptername\ \thechapter.\ #1}}{}} %

%\renewcommand{\sectionmark}[1]{         % Lower case Section marker style
%  \markright{\thesection.\ #1}}         %

\fancyhead[LE,RO]{\bfseries\thepage}    % Page number (boldface) in left on even
% pages and right on odd pages
\fancyhead[RE]{\bfseries\nouppercase{\leftmark}}      % Chapter in the right on even pages
\fancyhead[LO]{\bfseries\nouppercase{\rightmark}}     % Section in the left on odd pages

\let\headruleORIG\headrule
\renewcommand{\headrule}{\color{black} \headruleORIG}
\renewcommand{\headrulewidth}{1.0pt}
\usepackage{colortbl}
\arrayrulecolor{black}

\fancypagestyle{plain}{
  \fancyhead{}
  \fancyfoot{}
  \renewcommand{\headrulewidth}{0pt}
}

%\usepackage{algorithm}
%\usepackage[noend]{algorithmic}

%%% Clear Header %%%%%%%%%%%%%%%%%%%%%%%%%%%%%%%%%%%%%%%%%%%%%%%%%%%%%%%%%%%%%%%%%%
% Clear Header Style on the Last Empty Odd pages
\makeatletter

\def\cleardoublepage{\clearpage\if@twoside \ifodd\c@page\else%
  \hbox{}%
  \thispagestyle{empty}%              % Empty header styles
  %\newpage%
  \if@twocolumn\hbox{}\newpage\fi\fi\fi}

\makeatother
 
%%%%%%%%%%%%%%%%%%%%%%%%%%%%%%%%%%%%%%%%%%%%%%%%%%%%%%%%%%%%%%%%%%%%%%%%%%%%%%% 
% Prints your review date and 'Draft Version' (From Josullvn, CS, CMU)
\newcommand{\reviewtimetoday}[2]{\special{!userdict begin
    /bop-hook{gsave 20 710 translate 45 rotate 0.8 setgray
      /Times-Roman findfont 12 scalefont setfont 0 0   moveto (#1) show
      0 -12 moveto (#2) show grestore}def end}}
% You can turn on or off this option.
% \reviewtimetoday{\today}{Draft Version}
%%%%%%%%%%%%%%%%%%%%%%%%%%%%%%%%%%%%%%%%%%%%%%%%%%%%%%%%%%%%%%%%%%%%%%%%%%%%%%% 

\newenvironment{maxime}[1]
{
\vspace*{0cm}
\hfill
\begin{minipage}{0.5\textwidth}%
%\rule[0.5ex]{\textwidth}{0.1mm}\\%
\hrulefill $\:$ {\bf #1}\\
%\vspace*{-0.25cm}
\it 
}%
{%

\hrulefill
\vspace*{0.5cm}%
\end{minipage}
}

\let\minitocORIG\minitoc
\renewcommand{\minitoc}{\minitocORIG \vspace{1.5em}}

\usepackage{multirow}
%\usepackage{slashbox}

\newenvironment{bulletList}%
{ \begin{list}%
	{$\bullet$}%
	{\setlength{\labelwidth}{25pt}%
	 \setlength{\leftmargin}{30pt}%
	 \setlength{\itemsep}{\parsep}}}%
{ \end{list} }

\newtheorem{definition}{D�finition}
\renewcommand{\epsilon}{\varepsilon}

% centered page environment

\newenvironment{vcenterpage}
{\newpage\vspace*{\fill}\thispagestyle{empty}\renewcommand{\headrulewidth}{0pt}}
{\vspace*{\fill}}

%        LOADING TIKZ LIBRARIES
\usepackage[dvipsnames]{xcolor}
\usepackage{color}
\usepackage{tikz}
\usepackage{xfrac}
\usetikzlibrary{backgrounds}
\usetikzlibrary{mindmap}
\definecolor{mycolor}{RGB}{160,160,160}
\definecolor{mycolor1}{RGB}{192,192,192}

%     Theorem, definition, lemma, ...
\usepackage{amsmath,amsfonts,amssymb,amsthm,epsfig,epstopdf,titling,url,array}

\theoremstyle{plain}
\newtheorem{thm}{Veta}[section]
\newtheorem{lem}[thm]{Lema}
\newtheorem{prop}[thm]{Tvrdenie}
\newtheorem*{cor}{Z�ver}

\theoremstyle{definition}
\newtheorem{defn}{Defin\'{i}cia}[section]
\newtheorem{conj}{Predpoklad}[section]
\newtheorem{exmp}{Pr\'{i}klad}[section]

\theoremstyle{remark}
\newtheorem{rem}{Koment\'{a}r}
\newtheorem*{rem1}{Koment\'{a}r}
\newtheorem*{note}{Pozn\'{a}mka}

% pre zvyraznenie zdrojoveho kodu

\usepackage{lscape}        % otoci rozlozenie tabulky
\usepackage{listings}      % pre zdrojove kody
\usepackage{xcolor}        % pre definovnie vlastnej farby

\definecolor{mygray}{rgb}{0.97,0.97,0.97}

% Listings caption title
\renewcommand{\lstlistingname}{Code}

% New environment for listings
\lstnewenvironment{codelist}[2]
{\lstset{language=#1,
         aboveskip=10pt,
         belowskip=10pt,
         basicstyle=\scriptsize,
         frame=single,  % none, single or leftline
         framesep=4pt,
         backgroundcolor=\color{mygray},
         showstringspaces=true,
         caption=#2,
         captionpos=b}
}
{}

% New environment for listings
\lstnewenvironment{codelistwhite}[2]
{\lstset{language=#1,
         aboveskip=10pt,
         belowskip=10pt,
         basicstyle=\scriptsize,
         frame=none,  % none, single or leftline
         framesep=4pt,
         backgroundcolor=\color{white},
         showstringspaces=true,
         caption=#2,
         captionpos=b}
}
{}

